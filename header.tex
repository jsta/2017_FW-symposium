%% header.tex
%%
%% Copyright (C) 2016 - 2017  Dirk Eddelbuettel
%%
%% This file is part of samples-rmarkdown-metropolis repository.
%%
%% samples-rmarkdown-metropolis is free software: you can redistribute it
%% and/or modify it under the terms of the GNU General Public License as
%% published by the Free Software Foundation, either version 2 of the
%% License, or (at your option) any later version.
%%
%% samples-rmarkdown-metropolis is distributed in the hope that it will be
%% useful, but WITHOUT ANY WARRANTY; without even the implied warranty of
%% MERCHANTABILITY or FITNESS FOR A PARTICULAR PURPOSE.  See the GNU General
%% Public License for more details.
%%
%% You should have received a copy of the GNU General Public License along with
%% samples-rmarkdown-metropolis.  If not, see <http://www.gnu.org/licenses/>.

%% If you have the Fira font installed, to actually have it used it 
%% via rmarkdown you need to declare it here 
%\setsansfont[ItalicFont={Fira Sans Light Italic},BoldFont={Fira Sans},BoldItalicFont={Fira Sans Italic}]{Fira Sans Light}
%\setmonofont[BoldFont={Fira Mono Medium}]{Fira Mono}

%% You can set various Metropolis options via \metroset{} here
%\metroset{....}

%% You can redefine colours, mostly by borrowing from Beamer
\setbeamercolor{frametitle}{bg=gray}

%% You also use hyperref, and pick colors 
\hypersetup{colorlinks,citecolor=orange,filecolor=red,linkcolor=brown,urlcolor=blue}

%% when rendered with rmarkdown, somehow the unicode char for the dot
%% disappears so we redefine it here -- that is an older comments, seems font-specific
%\renewcommand{\textbullet}{$\cdot$}
%\renewcommand{\itemBullet}{▸}   % unicode U+25b8 'black right pointing small triangle'

%% The institute macro puts a small line for affiliation at the bottom
%\institute{Institute of Institutionalism} 

%% We can also place a logo
%\titlegraphic{\hfill\includegraphics[height=1cm]{someLogo.pdf}}

%%% Local Variables:
%%% mode: latex
%%% TeX-master: t
%%% End:

\def\begincols{\begin{columns}}
\def\endcols{\end{columns}}
\def\begincol{\begin{column}}
\def\endcol{\end{column}}